\documentclass[]{article}
\usepackage{lmodern}
\usepackage{amssymb,amsmath}
\usepackage{ifxetex,ifluatex}
\usepackage{fixltx2e} % provides \textsubscript
\ifnum 0\ifxetex 1\fi\ifluatex 1\fi=0 % if pdftex
  \usepackage[T1]{fontenc}
  \usepackage[utf8]{inputenc}
\else % if luatex or xelatex
  \ifxetex
    \usepackage{mathspec}
  \else
    \usepackage{fontspec}
  \fi
  \defaultfontfeatures{Ligatures=TeX,Scale=MatchLowercase}
\fi
% use upquote if available, for straight quotes in verbatim environments
\IfFileExists{upquote.sty}{\usepackage{upquote}}{}
% use microtype if available
\IfFileExists{microtype.sty}{%
\usepackage{microtype}
\UseMicrotypeSet[protrusion]{basicmath} % disable protrusion for tt fonts
}{}
\usepackage[margin=1in]{geometry}
\usepackage{hyperref}
\hypersetup{unicode=true,
            pdftitle={log mining},
            pdfauthor={Holly Beale},
            pdfborder={0 0 0},
            breaklinks=true}
\urlstyle{same}  % don't use monospace font for urls
\usepackage{color}
\usepackage{fancyvrb}
\newcommand{\VerbBar}{|}
\newcommand{\VERB}{\Verb[commandchars=\\\{\}]}
\DefineVerbatimEnvironment{Highlighting}{Verbatim}{commandchars=\\\{\}}
% Add ',fontsize=\small' for more characters per line
\usepackage{framed}
\definecolor{shadecolor}{RGB}{248,248,248}
\newenvironment{Shaded}{\begin{snugshade}}{\end{snugshade}}
\newcommand{\AlertTok}[1]{\textcolor[rgb]{0.94,0.16,0.16}{#1}}
\newcommand{\AnnotationTok}[1]{\textcolor[rgb]{0.56,0.35,0.01}{\textbf{\textit{#1}}}}
\newcommand{\AttributeTok}[1]{\textcolor[rgb]{0.77,0.63,0.00}{#1}}
\newcommand{\BaseNTok}[1]{\textcolor[rgb]{0.00,0.00,0.81}{#1}}
\newcommand{\BuiltInTok}[1]{#1}
\newcommand{\CharTok}[1]{\textcolor[rgb]{0.31,0.60,0.02}{#1}}
\newcommand{\CommentTok}[1]{\textcolor[rgb]{0.56,0.35,0.01}{\textit{#1}}}
\newcommand{\CommentVarTok}[1]{\textcolor[rgb]{0.56,0.35,0.01}{\textbf{\textit{#1}}}}
\newcommand{\ConstantTok}[1]{\textcolor[rgb]{0.00,0.00,0.00}{#1}}
\newcommand{\ControlFlowTok}[1]{\textcolor[rgb]{0.13,0.29,0.53}{\textbf{#1}}}
\newcommand{\DataTypeTok}[1]{\textcolor[rgb]{0.13,0.29,0.53}{#1}}
\newcommand{\DecValTok}[1]{\textcolor[rgb]{0.00,0.00,0.81}{#1}}
\newcommand{\DocumentationTok}[1]{\textcolor[rgb]{0.56,0.35,0.01}{\textbf{\textit{#1}}}}
\newcommand{\ErrorTok}[1]{\textcolor[rgb]{0.64,0.00,0.00}{\textbf{#1}}}
\newcommand{\ExtensionTok}[1]{#1}
\newcommand{\FloatTok}[1]{\textcolor[rgb]{0.00,0.00,0.81}{#1}}
\newcommand{\FunctionTok}[1]{\textcolor[rgb]{0.00,0.00,0.00}{#1}}
\newcommand{\ImportTok}[1]{#1}
\newcommand{\InformationTok}[1]{\textcolor[rgb]{0.56,0.35,0.01}{\textbf{\textit{#1}}}}
\newcommand{\KeywordTok}[1]{\textcolor[rgb]{0.13,0.29,0.53}{\textbf{#1}}}
\newcommand{\NormalTok}[1]{#1}
\newcommand{\OperatorTok}[1]{\textcolor[rgb]{0.81,0.36,0.00}{\textbf{#1}}}
\newcommand{\OtherTok}[1]{\textcolor[rgb]{0.56,0.35,0.01}{#1}}
\newcommand{\PreprocessorTok}[1]{\textcolor[rgb]{0.56,0.35,0.01}{\textit{#1}}}
\newcommand{\RegionMarkerTok}[1]{#1}
\newcommand{\SpecialCharTok}[1]{\textcolor[rgb]{0.00,0.00,0.00}{#1}}
\newcommand{\SpecialStringTok}[1]{\textcolor[rgb]{0.31,0.60,0.02}{#1}}
\newcommand{\StringTok}[1]{\textcolor[rgb]{0.31,0.60,0.02}{#1}}
\newcommand{\VariableTok}[1]{\textcolor[rgb]{0.00,0.00,0.00}{#1}}
\newcommand{\VerbatimStringTok}[1]{\textcolor[rgb]{0.31,0.60,0.02}{#1}}
\newcommand{\WarningTok}[1]{\textcolor[rgb]{0.56,0.35,0.01}{\textbf{\textit{#1}}}}
\usepackage{graphicx,grffile}
\makeatletter
\def\maxwidth{\ifdim\Gin@nat@width>\linewidth\linewidth\else\Gin@nat@width\fi}
\def\maxheight{\ifdim\Gin@nat@height>\textheight\textheight\else\Gin@nat@height\fi}
\makeatother
% Scale images if necessary, so that they will not overflow the page
% margins by default, and it is still possible to overwrite the defaults
% using explicit options in \includegraphics[width, height, ...]{}
\setkeys{Gin}{width=\maxwidth,height=\maxheight,keepaspectratio}
\IfFileExists{parskip.sty}{%
\usepackage{parskip}
}{% else
\setlength{\parindent}{0pt}
\setlength{\parskip}{6pt plus 2pt minus 1pt}
}
\setlength{\emergencystretch}{3em}  % prevent overfull lines
\providecommand{\tightlist}{%
  \setlength{\itemsep}{0pt}\setlength{\parskip}{0pt}}
\setcounter{secnumdepth}{0}
% Redefines (sub)paragraphs to behave more like sections
\ifx\paragraph\undefined\else
\let\oldparagraph\paragraph
\renewcommand{\paragraph}[1]{\oldparagraph{#1}\mbox{}}
\fi
\ifx\subparagraph\undefined\else
\let\oldsubparagraph\subparagraph
\renewcommand{\subparagraph}[1]{\oldsubparagraph{#1}\mbox{}}
\fi

%%% Use protect on footnotes to avoid problems with footnotes in titles
\let\rmarkdownfootnote\footnote%
\def\footnote{\protect\rmarkdownfootnote}

%%% Change title format to be more compact
\usepackage{titling}

% Create subtitle command for use in maketitle
\providecommand{\subtitle}[1]{
  \posttitle{
    \begin{center}\large#1\end{center}
    }
}

\setlength{\droptitle}{-2em}

  \title{log mining}
    \pretitle{\vspace{\droptitle}\centering\huge}
  \posttitle{\par}
    \author{Holly Beale}
    \preauthor{\centering\large\emph}
  \postauthor{\par}
      \predate{\centering\large\emph}
  \postdate{\par}
    \date{16 August, 2019}


\begin{document}
\maketitle

\begin{Shaded}
\begin{Highlighting}[]
\KeywordTok{library}\NormalTok{(tidyverse)}
\end{Highlighting}
\end{Shaded}

\begin{verbatim}
## Registered S3 methods overwritten by 'ggplot2':
##   method         from 
##   [.quosures     rlang
##   c.quosures     rlang
##   print.quosures rlang
\end{verbatim}

\begin{verbatim}
## Registered S3 method overwritten by 'rvest':
##   method            from
##   read_xml.response xml2
\end{verbatim}

\begin{verbatim}
## -- Attaching packages ------------------------------------------------------------- tidyverse 1.2.1 --
\end{verbatim}

\begin{verbatim}
## v ggplot2 3.1.1       v purrr   0.3.2  
## v tibble  2.1.1       v dplyr   0.8.0.1
## v tidyr   0.8.3       v stringr 1.4.0  
## v readr   1.3.1       v forcats 0.4.0
\end{verbatim}

\begin{verbatim}
## -- Conflicts ---------------------------------------------------------------- tidyverse_conflicts() --
## x dplyr::filter() masks stats::filter()
## x dplyr::lag()    masks stats::lag()
\end{verbatim}

\begin{Shaded}
\begin{Highlighting}[]
\KeywordTok{library}\NormalTok{(forcats)}
\KeywordTok{library}\NormalTok{(knitr)}
\KeywordTok{library}\NormalTok{(stringi)}
\end{Highlighting}
\end{Shaded}

\hypertarget{section}{%
\section{}\label{section}}

\begin{Shaded}
\begin{Highlighting}[]
\NormalTok{metrics_to_extract <-}\StringTok{ }\KeywordTok{c}\NormalTok{(}\StringTok{"fastq R1 linecount"}\NormalTok{, }
  \StringTok{"fastq R2 linecount"}\NormalTok{,}
  \StringTok{"fastq R1 md5sum"}\NormalTok{, }
  \StringTok{"fastq R2 md5sum"}\NormalTok{, }
  \StringTok{"kallisto results md5sum"}\NormalTok{,}
  \StringTok{"rsem results md5sum"}\NormalTok{,}
  \StringTok{"cutadapt R1 linecount"}\NormalTok{,}
  \StringTok{"cutadapt R2 linecount"}\NormalTok{,}
  \StringTok{"cutadapt R1 md5sum"}\NormalTok{,}
  \StringTok{"cutadapt R2 md5sum"}\NormalTok{,}
  \StringTok{"rsem results md5sum"}\NormalTok{,}
  \StringTok{"star bam md5sum"}\NormalTok{)}

\NormalTok{branches_analzyed <-}\StringTok{  }\KeywordTok{c}\NormalTok{(}\StringTok{"^ungroomed"}\NormalTok{, }\StringTok{"^sorted"}\NormalTok{, }\StringTok{"^groomed"}\NormalTok{)}
\NormalTok{measureable_values <-}\StringTok{ }\KeywordTok{c}\NormalTok{(}\StringTok{"linecount"}\NormalTok{, }\StringTok{"md5sum"}\NormalTok{)}
\end{Highlighting}
\end{Shaded}

\begin{Shaded}
\begin{Highlighting}[]
\NormalTok{data_dir <-}\StringTok{  "~/Downloads/seqgroom-jackie-holly/"}

\NormalTok{summary_logs <-}\StringTok{  }\KeywordTok{list.files}\NormalTok{(data_dir, }\DataTypeTok{pattern=}\StringTok{".*-SUMMARY.log"}\NormalTok{, }\DataTypeTok{recursive =} \OtherTok{TRUE}\NormalTok{)}

\NormalTok{logged_measurements <-}\StringTok{ }\KeywordTok{lapply}\NormalTok{(summary_logs, }\ControlFlowTok{function}\NormalTok{(log_name) \{}
  \CommentTok{# log_name <- summary_logs[1]}
\NormalTok{  log_contents_raw <-}\StringTok{ }\KeywordTok{tibble}\NormalTok{(}
    \DataTypeTok{file_name =}\NormalTok{ log_name,}
    \DataTypeTok{sample_name =} \KeywordTok{gsub}\NormalTok{(}\StringTok{"-SUMMARY.log"}\NormalTok{, }\StringTok{""}\NormalTok{, }\KeywordTok{basename}\NormalTok{(file_name)),}
    \DataTypeTok{full_line =} \KeywordTok{scan}\NormalTok{(}\KeywordTok{paste0}\NormalTok{(data_dir, log_name), }\DataTypeTok{what=}\StringTok{"list"}\NormalTok{, }\DataTypeTok{sep=}\StringTok{"}\CharTok{\textbackslash{}n}\StringTok{"}\NormalTok{)}
\NormalTok{  ) }\OperatorTok
\StringTok{    }\KeywordTok{filter}\NormalTok{(}\KeywordTok{grepl}\NormalTok{(}\KeywordTok{paste}\NormalTok{(metrics_to_extract, }\DataTypeTok{collapse=}\StringTok{"|"}\NormalTok{), full_line))}
\NormalTok{  log_contents_raw }\OperatorTok
\StringTok{    }\KeywordTok{mutate}\NormalTok{(}
      \CommentTok{#      date_stamp = gsub("2019 .*$", "2019", full_line)}
      \DataTypeTok{measurement =} \KeywordTok{gsub}\NormalTok{(}\StringTok{"^.*2019 "}\NormalTok{, }\StringTok{""}\NormalTok{, full_line),}
      \CommentTok{#      value = gsub("^.*: ([a-z0-9]*) .*$", "\textbackslash{}\textbackslash{}1", full_line),}
      \DataTypeTok{measurement_name =} \KeywordTok{gsub}\NormalTok{(}\StringTok{":.*$"}\NormalTok{, }\StringTok{""}\NormalTok{, measurement),}
      \DataTypeTok{value =} \KeywordTok{gsub}\NormalTok{(}\StringTok{"[[:space:]] .*"}\NormalTok{, }\StringTok{""}\NormalTok{, }\KeywordTok{gsub}\NormalTok{(}\StringTok{"^.*:"}\NormalTok{, }\StringTok{""}\NormalTok{,full_line))}
\NormalTok{    ) }\OperatorTok
\StringTok{    }\NormalTok{rowwise }\OperatorTok
\StringTok{    }\KeywordTok{mutate}\NormalTok{(}
      \DataTypeTok{value_measured =}\NormalTok{ measureable_values[}\KeywordTok{stri_detect_fixed}\NormalTok{(measurement_name, measureable_values)],}
      \DataTypeTok{analysis_branch =} \KeywordTok{gsub}\NormalTok{(}\StringTok{"}\CharTok{\textbackslash{}\textbackslash{}}\StringTok{^"}\NormalTok{, }
                             \StringTok{""}\NormalTok{,}
\NormalTok{                             branches_analzyed}
\NormalTok{                             [}\KeywordTok{stri_detect_regex}\NormalTok{(measurement_name, branches_analzyed)]}
\NormalTok{      ),}
      \DataTypeTok{output_measured =} \KeywordTok{gsub}\NormalTok{(analysis_branch, }\StringTok{""}\NormalTok{, }\KeywordTok{gsub}\NormalTok{(value_measured, }\StringTok{""}\NormalTok{, measurement_name))}
\NormalTok{    ) }\OperatorTok\StringTok{ }\KeywordTok{select}\NormalTok{(sample_name, analysis_branch, output_measured, value_measured, value)}
\NormalTok{\}) }\OperatorTok\StringTok{ }\NormalTok{bind_rows}

\NormalTok{logged_measurements }\OperatorTok\StringTok{ }\KeywordTok{group_by}\NormalTok{(sample_name, output_measured, value_measured,) }\OperatorTok
\StringTok{  }\KeywordTok{summarize}\NormalTok{(}\DataTypeTok{distinct_values =} \KeywordTok{length}\NormalTok{(}\KeywordTok{unique}\NormalTok{(value)))}
\end{Highlighting}
\end{Shaded}

\begin{verbatim}
## Warning: Grouping rowwise data frame strips rowwise nature
\end{verbatim}

\begin{verbatim}
## # A tibble: 187 x 4
## # Groups:   sample_name, output_measured [119]
##    sample_name output_measured      value_measured distinct_values
##    <chr>       <chr>                <chr>                    <int>
##  1 CCLE_BLCA   " cutadapt R1 "      linecount                    2
##  2 CCLE_BLCA   " cutadapt R1 "      md5sum                       3
##  3 CCLE_BLCA   " cutadapt R2 "      linecount                    2
##  4 CCLE_BLCA   " cutadapt R2 "      md5sum                       3
##  5 CCLE_BLCA   " fastq R1 "         linecount                    1
##  6 CCLE_BLCA   " fastq R1 "         md5sum                       3
##  7 CCLE_BLCA   " fastq R2 "         linecount                    1
##  8 CCLE_BLCA   " fastq R2 "         md5sum                       3
##  9 CCLE_BLCA   " kallisto results " md5sum                       2
## 10 CCLE_BLCA   " rsem results "     md5sum                       2
## # ... with 177 more rows
\end{verbatim}

\begin{Shaded}
\begin{Highlighting}[]
\NormalTok{logged_measurements }\OperatorTok\StringTok{ }\KeywordTok{filter}\NormalTok{(}\KeywordTok{grepl}\NormalTok{(}\StringTok{"cutadapt R1"}\NormalTok{, output_measured),}
\NormalTok{                               value_measured }\OperatorTok{==}\StringTok{ "linecount"}\NormalTok{,}
\NormalTok{                               sample_name }\OperatorTok{==}\StringTok{ "CCLE_BLCA"}\NormalTok{)}
\end{Highlighting}
\end{Shaded}

\begin{verbatim}
## Source: local data frame [3 x 5]
## Groups: <by row>
## 
## # A tibble: 3 x 5
##   sample_name analysis_branch output_measured value_measured value       
##   <chr>       <chr>           <chr>           <chr>          <chr>       
## 1 CCLE_BLCA   ungroomed       " cutadapt R1 " linecount      " 0"        
## 2 CCLE_BLCA   sorted          " cutadapt R1 " linecount      " 159942736"
## 3 CCLE_BLCA   groomed         " cutadapt R1 " linecount      " 159942736"
\end{verbatim}

\hypertarget{line-count-plots}{%
\section{Line count plots}\label{line-count-plots}}

\begin{Shaded}
\begin{Highlighting}[]
\NormalTok{cutadapt_linecounts <-}\StringTok{ }\NormalTok{logged_measurements }\OperatorTok\StringTok{ }\KeywordTok{filter}\NormalTok{(}\KeywordTok{grepl}\NormalTok{(}\StringTok{"cutadapt R1"}\NormalTok{, output_measured),}
\NormalTok{                               value_measured }\OperatorTok{==}\StringTok{ "linecount"}\NormalTok{)}

\NormalTok{cutadapt_linecounts}\OperatorTok{$}\NormalTok{analysis_branch <-}\StringTok{ }\KeywordTok{factor}\NormalTok{(cutadapt_linecounts}\OperatorTok{$}\NormalTok{analysis_branch, }\DataTypeTok{levels =}  \KeywordTok{c}\NormalTok{(}\StringTok{"ungroomed"}\NormalTok{, }\StringTok{"sorted"}\NormalTok{, }\StringTok{"groomed"}\NormalTok{))}


\KeywordTok{ggplot}\NormalTok{(cutadapt_linecounts) }\OperatorTok{+}\StringTok{ }\KeywordTok{geom_point}\NormalTok{(}\KeywordTok{aes}\NormalTok{(}\DataTypeTok{x=}\NormalTok{analysis_branch, }\DataTypeTok{y=}\KeywordTok{as.numeric}\NormalTok{(value), }\DataTypeTok{color =}\NormalTok{ analysis_branch)) }\OperatorTok{+}\StringTok{ }\KeywordTok{facet_wrap}\NormalTok{(}\OperatorTok{~}\NormalTok{sample_name, }\DataTypeTok{nrow=}\DecValTok{1}\NormalTok{)  }\OperatorTok{+}
\StringTok{  }\KeywordTok{theme}\NormalTok{(}\DataTypeTok{axis.text.x =} \KeywordTok{element_blank}\NormalTok{()) }\OperatorTok{+}
\StringTok{ }\KeywordTok{theme}\NormalTok{(}\DataTypeTok{strip.text.x =} \KeywordTok{element_text}\NormalTok{(}\DataTypeTok{angle =} \DecValTok{90}\NormalTok{))}
\end{Highlighting}
\end{Shaded}

\includegraphics{log_mining_files/figure-latex/unnamed-chunk-4-1.pdf}


\end{document}
